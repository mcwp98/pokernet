\#\+Pokernet The pokernet game requires python and node.\+js function. The server end uses the django framework for the general application and a node.\+js instance running to handle games and logic. Several dependencies are also required on the server, including django for python and several libraries for node.\+js

\subsection*{Install}

\subsubsection*{Python}

Python can be downloaded and installed to your system from \href{https://www.python.org/}{\tt https\+://www.\+python.\+org/}

\paragraph*{Django}

Django may be installed on the server by using $>$git clone git\+://github.com/django/django.\+git django-\/trunk

and then

$>$sudo pip install -\/e django-\/trunk/

\paragraph*{Django Application Dependencies}

We require the Pillow module, to install\+: \begin{quote}
pip install Pillow \end{quote}


\subsubsection*{Node.\+js}

Node.\+js can be downloaded and installed to your system from \href{http://nodejs.org/}{\tt http\+://nodejs.\+org/}

\paragraph*{Node.\+js Applications}

Pokernet uses several node.\+js applications including express, handranker, and socket.\+io. to instal them, use npm\+: $>$npm -\/g install express handranker socket.\+io

then cd to the /sockets/ directory and run

$>$npm link expres handranker socket.\+io

\subsubsection*{lite\+S\+Q\+L Database}

pokernet currently utilizes the lite\+S\+Q\+L database, to create, navigate to the application directory and run\+: $>$python manage.\+py syncdb

\subsection*{Development}

To run both server and client locally during development, a script is provided that will run both the node.\+js and django dev server. To use, simply cd into the project directory and run\+: \begin{quote}
./start.sh \end{quote}


\subsection*{Deployment}

Various deployment strategies exist, however, note that the current version should not be used on a production server as many security features have not been implemented.

\subsubsection*{Django with W\+S\+G\+I}

Information on deploying the Django application on your server with W\+S\+G\+I can be found at \href{https://docs.djangoproject.com/en/dev/howto/deployment/wsgi/}{\tt https\+://docs.\+djangoproject.\+com/en/dev/howto/deployment/wsgi/}

\subsubsection*{Node.\+js}

Not much has to be done to deploy the node.\+js server. You may simply run\+: $>$forever node sockets/server.\+js 